% Section 3 - Launch files
% Roberto Masocco <roberto.masocco@uniroma2.it>
% May 21, 2024

% ### Launch files ###
\section{Launch files}

% --- Scripting ROS 2 architectures ---
\begin{frame}{Scripting ROS 2 architectures}
	A ROS 2-based control architecture for a robot can easily get to have 20 nodes or more.\\
	\bigskip
	It then becomes critical to be able to \textbg{automate startup and configuration} of all the modules, or some subsets, also for testing.\\
	\bigskip
	That is what the ROS 2 \textbg{Launch System} is for.
\end{frame}

% --- Launch files ---
\begin{frame}{Launch files}
	\textbg{Launch files} are \textbg{Python scripts} that specify how ROS 2 modules must be \textbg{located}, \textbg{configured} and \textbg{started}. Their format is such that the Launch System can parse and integrate them when invoked.\\\vspace{10pt}
	Many things can be configured about ROS 2 modules in such files:
	\begin{itemize}
		\item console and text files \textbg{logs};
		\item command line \textbg{arguments};
		\item node \textbg{parameters};
		\item \textbg{remappings} of namespaces, node names, topics, services and actions.
	\end{itemize}
	\vspace{10pt}
	It is also possible to start \textbg{custom executables}, define \textbg{environment variables}, and more.\\\vspace{10pt}
	Launch files may be \textbg{included}, so that \textbg{large architectures} can be started with \textbg{one command}.
\end{frame}
\begin{frame}[fragile]{Launch files}
	\begin{columns}\column{.9\textwidth}
		\begin{lstlisting}[language=Python, caption=Minimal example of a launch file that starts a ROS 2 node.]
from launch import LaunchDescription
from launch_ros.actions import Node

# The following function MUST be specified

def generate_launch_description():
    """Builds a launch description."""
    ld = LaunchDescription()
    node = Node(
      package='PACKAGE_NAME',
      executable='EXECUTABLE_NAME')
    ld.add_action(node)
    return ld
\end{lstlisting}
	\end{columns}
\end{frame}

% --- Launch System CLI commands ---
\begin{frame}{Launch System CLI commands}
  The main command to \textbg{execute the \texttt{LaunchDescription}} specified in a launch file is\\
  \bigskip
  \texttt{ros2 launch PACKAGE\_NAME LAUNCH\_FILE}\\
  \bigskip
  This will start a \textbg{Python interpreter} in your current shell, that will \textbg{parse} the launch file and \textbg{execute} the \texttt{LaunchDescription}. You will \textbg{interface with the underlying processes} through this interpreter instance. You can also specify \textbg{launch arguments} that you will be able to access from within the launch file:\\
  \bigskip
  \texttt{ros2 launch PACKAGE\_NAME LAUNCH\_FILE ARG1:=VALUE1 ARG2:=VALUE2 ...}\\
  \bigskip
  Have a look at the options of this command!
\end{frame}

% --- Coding launch files ---
\begin{frame}{Coding launch files}{Tips and best practices}
	\begin{itemize}
		\item Their \textbg{extension} is usually \texttt{.launch.py}.
		\item They are usually placed in a package subdirectory named \texttt{launch/} that is \textbg{installed} in the workspace path during build, via appropriate directives in either \texttt{CMakeLists.txt} or \texttt{setup.py} files.
		\item A module can have its own launch files but those for the entire \textbg{architecture} must form an appropriate \textbg{package}, whose name is usually \texttt{PROJECT\_bringup}.
	\end{itemize}
	\begin{block}{}
		\centering
		Find a comprehensive description of all the features of launch files in \href{https://github.com/IntelligentSystemsLabUTV/ros2-examples/blob/humble/launch_files.md}{\color{blue}\underline{\texttt{launch\_files.md}}}.
	\end{block}
\end{frame}

% --- Example: Bringup package ---
\begin{frame}{Example: Bringup package}
  \begin{block}{}
    \centering
	  Now go have a look at the \href{https://github.com/IntelligentSystemsLabUTV/ros2-examples/tree/humble/src/ros2_examples_bringup}{\color{blue}\underline{\texttt{ros2-examples/src/ros2\_examples\_bringup}}} package!
  \end{block}
\end{frame}
